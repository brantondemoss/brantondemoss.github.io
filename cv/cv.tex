% LaTeX resume using res.cls
\documentclass[margin]{res}
\usepackage{mathtools} 
\usepackage{amsmath}
%\usepackage{helvetica} % uses helvetica postscript font (download helvetica.sty)
%\usepackage{newcent}   % uses new century schoolbook postscript font 
\setlength{\textwidth}{5.1in} % set width of text portion

\begin{document}

% Center the name over the entire width of resume:

\moveleft.5\hoffset\centerline{\large\bf Branton DeMoss}
%\vspace{10px}
%\moveleft.5\hoffset\centerline{branton.demoss@eng.ox.ac.uk \(|\) brantondemoss.com}
%\moveleft.5\hoffset\centerline{ \begin{ncolumn}{3}
%     & branton.demoss@eng.ox.ac.uk & \qquad St Edmund Hall \\
%     & brantondemoss.com & \qquad Queen's Lane, Oxford \\
%    \large\bf Branton DeMoss & +1-720-592-5911 & \qquad OX1 4AR, UK   \\
%  \end{ncolumn}}
% Draw a horizontal line the whole width of resume:
% \moveleft\hoffset\vbox{\hrule width\resumewidth height 1pt}\smallskip
% address begins here
% Again, the address lines must be centered over entire width of resume:




\begin{resume}
  \section{SUMMARY}  Interested in the intersection of classical planning
  and deep-learning based world modeling to build autonomous agents that
  can think ahead to act in the world.

  \section{CONTACT} \begin{ncolumn}{2}
     branton.demoss@eng.ox.ac.uk & \qquad St Edmund Hall \\
     brantondemoss.com & \qquad Queen's Lane, Oxford \\
     +1-720-592-5911 & \qquad OX1 4AR, UK   \\                   
  \end{ncolumn}
  
 

  \section{EDUCATION} {\sl DPhil Candidate in Artificial Intelligence } \hfill 2021-\\
                      % \sl will be bold italic in New Century Schoolbook (or
                      % any postscript font) and just slanted in
                      % Computer Modern (default) font
                University of Oxford  \\


  {\sl BA Mathematics and Physics} \hfill 2014-18\\
                University of Colorado Boulder  \\
 
\section{EXPERIENCE} The Collaboratory \hfill 2020- \\
                 {\sl Co-founder; Chief Science Officer}
                 \begin{itemize}  \itemsep -2pt %reduce space between items
                 \item Developed deep-learning based document embedder
                   based on language and graph structure, and related algorithms
                   for knowledge curation.
                 \item Led product strategy, ML R\&D,
                   and customer-informed design.
                 \item Admitted to Techstars class of 2021 ($<1\%$ applicants admitted)
                 \end{itemize}

                 Comma.ai \hfill            2020 \\
		{\sl ML Research Intern}
                 \begin{itemize}  \itemsep -2pt %reduce space between items
                 \item Extended	vision module architecture and ported recurrent
                   neural network for driving policy from Tensorflow to PyTorch.
                 \item Studied effects of new stochastic dynamics model on driving
                   policy quality.
                 \end{itemize}

 
               Front Range Geosciences \hfill            2017-20 \\
                 {\sl Research Scientist}
                 \begin{itemize}  \itemsep -2pt %reduce space between items
                 \item Developed convolutional neural network (CNN) to detect
                   seismic first break events. System now used in production at
                   multinational seismic exploration corporations.
                 \item Incorporated differentiable Gaussian mixture models in
                   deep learning system to model energy-time uncertainty in
                   wavelet arrival.
                 \item Developed Generative Adversarial Network (GAN) to pre-train
                   CNN when supervisory targets unavailable.
                 \item Wrote eikonal wave equation propagator (C++) for
                   psuedo-structured 3D meshes for tomographic seismic imaging.
                 \end{itemize} 
\section{RESEARCH} Center for Theory of Quantum Matter \hfill            2017 \\
		{\sl Research Assistant}
                 \begin{itemize}  \itemsep -2pt %reduce space between items
                 \item Characterized quantum many-body localization (MBL) under Floquet conditions.
                 \end{itemize}

                Mathematics Department, CU Boulder \hfill            2016 \\
		 {\sl Research Assistant}
                 \begin{itemize}  \itemsep -2pt %reduce space between items
                 \item Investigated knot-theoretic properties of topological quantum field theories.
                 \end{itemize}

                High Enery Particle Physics Group, Physics Department, CU Boulder  \hfill            2014-15 \\
		{\sl Research Assistant}
                 \begin{itemize}  \itemsep -2pt %reduce space between items
                 \item Characterized effects of beamline material geometry on particle
                   correlation statistics for the Deep Underground Neutrino Experiment
                   (DUNE).
                 \end{itemize}

\section{PUBLICATIONS} {\sl Combining physics and deep learning to automatically \hfill            $2021$ \\ pick first breaks in the Permian Basin} \\
		To appear in {\sl SEG Technical Program Expanded Abstracts}

                {\sl Ein Liebesbrief an KataGo} \hfill 2020 \\
                Deutsche Go Zeitung, Ausgabe 4/2020

                {\sl Love Letter to KataGo, or:\hfill $2020$\\ Go AI Past, Present, and Future} \\
                American Go E-Journal
                
                {\sl DeepTrace: A breakthrough application of deep learning \hfill $2019$\\ to automate first break picking}  \\
                SEG 2019 Lenovo Thought Leadership Series

                {\sl Topology and Knot Theory} \hfill 2016 \\
                Course notes for CU Boulder special topics course: \\
                \textit{``Topology, Knot Theory, and their applications in Physics and Chemistry''}
                
                {\sl Secondary Particle Showers from Hadron Absorber Interactions} \hfill 2016 \\
                Long Baseline Neutrino Facility (LBNF) / Deep Underground Neutrino Experiment (DUNE) Collaboration Documents

                
                \section{AWARDS}
                
                {\sl Research Studentship} \hfill Oxford, 2021 \\
                {\sl Stribic-Martin Scholarship} \hfill Boulder, 2017 \\
                        {\sl UROP Fellowship} \hfill Boulder, 2017 \\
                        {\sl Dawkins Fund Award} \hfill Oxford, 2016 \\
                        {\sl Gilman Scholarship} \hfill Oxford, 2016 \\
                        {\sl Esteemed Scholar Award} \hfill Boulder, 2014 \\




                

\end{resume}
\end{document}
B
