% LaTeX resume using res.cls
\documentclass[margin]{res}
\usepackage{mathtools}
\usepackage{amsmath}
\usepackage{hyperref}
%\usepackage{helvetica} % uses helvetica postscript font (download helvetica.sty)
%\usepackage{newcent}   % uses new century schoolbook postscript font
\setlength{\textwidth}{5.1in} % set width of text portion

\begin{document}

% Center the name over the entire width of resume:

\moveleft0.5\hoffset\centerline{\large\bf Branton DeMoss}
\vspace{10px}
%\moveleft.5\hoffset\centerline{branton.demoss@eng.ox.ac.uk \(|\) brantondemoss.com}
%\moveleft.5\hoffset\centerline{ \begin{ncolumn}{3}
%     & branton.demoss@eng.ox.ac.uk & \qquad St Edmund Hall \\
%     & brantondemoss.com & \qquad Queen's Lane, Oxford \\
%    \large\bf Branton DeMoss & +1-720-592-5911 & \qquad OX1 4AR, UK   \\
%  \end{ncolumn}}
% Draw a horizontal line the whole width of resume:
% \moveleft\hoffset\vbox{\hrule width\resumewidth height 1pt}\smallskip
% address begins here
% Again, the address lines must be centered over entire width of resume:
\section{Contact}
  \begin{ncolumn}{2}
     \href{mailto:bdemoss@robots.ox.ac.uk}{bdemoss@robots.ox.ac.uk} & \hfill St Edmund Hall \\
     \href{https://brantondemoss.com/}{www.brantondemoss.com} & \hfill Queen's Lane, Oxford \\
     \href{tel:+447926576225}{+44 (0)7926 576225} & \hfill OX1 4AR, UK
  \end{ncolumn}

\begin{resume}
%    \section{Summary}
%    Working to understand emergence through complexity, machine learning, \\and algorithmic information theory.


    \section{Education} {\sl DPhil Candidate in Machine Learning} \hfill 2021-25 (expected)\\
                      % \sl will be bold italic in New Century Schoolbook (or
                      % any postscript font) and just slanted in
                      % Computer Modern (default) font
                University of Oxford

  {\sl BA Mathematics and Physics} \hfill 2018\\
  University of Colorado Boulder

  {\sl Visitor Mathematical and Theoretical Physics} \hfill 2016-17\\
  University of Oxford

  \section{Experience}
  Mathematical Institute, University of Oxford \hfill 2025-\\
                 {\sl Postdoctoral Research Associate}
                 \begin{itemize}  \itemsep -2pt %reduce space between items
                 \item Research on the mathematical and computational foundations of AI.
                 \end{itemize}

  Oxford Robotics Institute \hfill 2021-25\\
                 {\sl Graduate Student Researcher}
                 \begin{itemize}  \itemsep -2pt %reduce space between items
                 \item Research in complexity, generalization, reinforcement learning, world models.
                 \end{itemize}

                The Collaboratory \hfill 2020-23 \\
                 {\sl Co-founder; Chief Science Officer}
                 \begin{itemize}  \itemsep -2pt %reduce space between items
                 \item Deep learning on language and graphs for knowledge curation.
                 \item Raised \$2M, led product strategy, design, and ML R\&D.
                 \end{itemize}

                 Comma.ai \hfill            2020 \\
		{\sl ML Research Intern}
                 \begin{itemize}  \itemsep -2pt %reduce space between items
                 \item Reinforcement learning for self-driving cars.
                 \end{itemize}

               Front Range Geosciences \hfill            2017-20 \\
                 {\sl Machine Learning Engineer}
                 \begin{itemize}  \itemsep -2pt %reduce space between items
                 \item Sole dev on ML system for seismic data, sold to multinational co.
                 \end{itemize}
%\section{RESEARCH}
                 Center for Theory of Quantum Matter \hfill            2017 \\
		{\sl Research Assistant}
                 \begin{itemize}  \itemsep -2pt %reduce space between items
                 \item Research on quantum many-body localization.
                 \end{itemize}

                Mathematics Department, CU Boulder \hfill            2016 \\
		 {\sl Research Assistant}
                 \begin{itemize}  \itemsep -2pt %reduce space between items
                 \item Knot theory and topological quantum field theory.
                 \end{itemize}

                High Enery Particle Physics Group, Physics Department, CU Boulder  \hfill            2014-15 \\
		{\sl Research Assistant}
                 \begin{itemize}  \itemsep -2pt %reduce space between items
                 \item High performance Monte Carlo simulations (C++) for DUNE experiment.
                 \end{itemize}

                 \section{Publications}

                 {\sl The Complexity Dynamics of Grokking \hfill $2025$}\\
                 Physica D: Nonlinear Phenomena

                 {\sl The Complexity Dynamics of Double Descent \hfill $2025$}\\
                 Work in progress.\\
                 I explain double descent in neural networks from a complexity perspective.

                 {\sl LUMOS: Language-Conditioned Imitation Learning with World Models \hfill $2024$}\\
                 ICRA 2025

                 {\sl DITTO: Offline Imitation Learning with World Models \hfill            $2023$}\\
                 Under submission to NeurIPS\\
                 ar$\chi$iv:2302.03086


                 {\sl Combining physics and deep learning to automatically \hfill            $2021$ \\ pick first breaks in the Permian Basin} \\
		First International Meeting for Applied Geoscience \& Energy

                {\sl Ein Liebesbrief an KataGo} \hfill 2020 \\
                Deutsche Go Zeitung, Ausgabe 4/2020

                {\sl Love Letter to KataGo, or:\hfill $2020$\\ Go AI Past, Present, and Future} \\
                American Go E-Journal

                {\sl DeepTrace: A breakthrough application of deep learning \hfill $2019$\\ to automate first break picking}  \\
                SEG 2019 Lenovo Thought Leadership Series

                {\sl Topology and Knot Theory} \hfill 2016 \\
                Course notes for CU Boulder special topics course: \\
                \textit{``Topology, Knot Theory, and their applications in Physics and Chemistry''}

                {\sl Secondary Particle Showers from Hadron Absorber Interactions} \hfill 2016 \\
                Deep Underground Neutrino Experiment (DUNE) Collaboration Documents


                \section{Teaching}
                        {\sl Physics of Information and Complexity \hfill $2024$}\\
                        Received highest possible marks for teaching performance.\\
                Oxford, HT 24

                {\sl Philosophy of Emergence \hfill $2024$}\\
                Received highest possible marks for teaching performance.\\
                Oxford, HT 24

                {\sl Topics in Reinforcement Learning \hfill $2023$}\\
                Received highest possible marks for teaching performance.\\
                Oxford, MT 23

                {\sl Rocket League Behaviour Cloning from Unlabelled Data \hfill $2023$}\\
                Supervised Master's Thesis, Oxford\\
                Student obtained highest marks, and secured funded DPhil position in Oxford.

                \section{Talks}
                {\sl 2\textsuperscript{nd} Symposium on Algorithmic Information Theory and Machine Learning\hfill $2025$}\\
                Talk on my discovery of complexity phase transitions in learning systems.
                \href{https://www.youtube.com/watch?v=6QjSH9ghUJs}{Link.}

                {\sl ICRA 2025, Robot Foundation Models Session\hfill $2025$}\\
                Talk on our work LUMOS, addressing reinforcement learning in world models.

                {\sl Harvard/Tufts, Levin Group\hfill $2025$}\\
                Invited talk on complexity dynamics to Michael Levin's computational biology group.
                \href{https://www.youtube.com/watch?v=gb0z8lV1i78}{Link.}

                {\sl Oxford, Department of Physics\hfill $2024$}\\
                Invited talk on complexity dynamics to Ard Louis's research group.

                {\sl Oxford, Department of Statistics\hfill $2024$}\\
                Invited talk on complexity and generalization to the RainML group.


                \section{Awards}

                {\sl AWS Lighthouse Scholarship (fully funded PhD)} \hfill Oxford, 2021 \\
                {\sl Stribic-Martin Scholarship} \hfill Boulder, 2017 \\
                {\sl UROP Fellowship} \hfill Boulder, 2017 \\
                {\sl Dawkins Fund Award} \hfill Oxford, 2016 \\
                {\sl Gilman Scholarship} \hfill Oxford, 2016 \\
                {\sl Esteemed Scholar Award} \hfill Boulder, 2014 \\


                \section{References}
                        {\sl Prof. Nick Hawes}\\
                        Professor of AI and Robotics, Oxford\\
                        Director, Oxford Robotics Institute\\
                        nickh@robots.ox.ac.uk

                        {\sl Prof. Ingmar Posner}\\
                        Professor of Applied AI, Oxford\\
                        Deputy Director, Oxford Robotics Institute\\
                        ingmar@robots.ox.ac.uk

                        {\sl Prof. Jakob Foerster}\\
                        Associate Professor, Oxford\\
                        jakob@robots.ox.ac.uk

                        {\sl Prof. Jared Tanner (supervisor from Oct 2025)}\\
                          Professor of the Mathematics of Information, Oxford\\
                          tanner@maths.ox.ac.uk

\end{resume}
\end{document}
