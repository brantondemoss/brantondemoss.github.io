% LaTeX resume using res.cls
\documentclass[margin]{res}
\usepackage{mathtools}
\usepackage{amsmath}
\usepackage{hyperref}
%\usepackage{helvetica} % uses helvetica postscript font (download helvetica.sty)
%\usepackage{newcent}   % uses new century schoolbook postscript font
\setlength{\textwidth}{5.1in} % set width of text portion

\begin{document}

% Center the name over the entire width of resume:

\moveleft0.5\hoffset\centerline{\large\bf Branton DeMoss}
\vspace{10px}
%\moveleft.5\hoffset\centerline{branton.demoss@eng.ox.ac.uk \(|\) brantondemoss.com}
%\moveleft.5\hoffset\centerline{ \begin{ncolumn}{3}
%     & branton.demoss@eng.ox.ac.uk & \qquad St Edmund Hall \\
%     & brantondemoss.com & \qquad Queen's Lane, Oxford \\
%    \large\bf Branton DeMoss & +1-720-592-5911 & \qquad OX1 4AR, UK   \\
%  \end{ncolumn}}
% Draw a horizontal line the whole width of resume:
 \moveleft\hoffset\vbox{\hrule width\resumewidth height 1pt}\smallskip
% address begins here
% Again, the address lines must be centered over entire width of resume:
\section{CONTACT}
  \begin{ncolumn}{2}
     \href{mailto:bdemoss@robots.ox.ac.uk}{bdemoss@robots.ox.ac.uk} & \hfill St Edmund Hall \\
     \href{https://brantondemoss.com/}{www.brantondemoss.com} & \hfill Queen's Lane, Oxford \\
     \href{tel:+447926576225}{+44 (0)7926 576225} & \hfill OX1 4AR, UK
  \end{ncolumn}

\begin{resume}
    \section{SUMMARY}
    Working at the intersection of reinforcement learning, world modeling, and planning to
    build autonomous agents that can think ahead to act in the world.


  \section{EDUCATION} {\sl DPhil Candidate in Artificial Intelligence } \hfill 2021-\\
                      % \sl will be bold italic in New Century Schoolbook (or
                      % any postscript font) and just slanted in
                      % Computer Modern (default) font
                University of Oxford

  {\sl BA Mathematics and Physics} \hfill 2018\\
  University of Colorado Boulder

\section{EXPERIENCE} Oxford Robotics Institute \hfill 2021- \\
                 {\sl Graduate Student Researcher}
                 \begin{itemize}  \itemsep -2pt %reduce space between items
                 \item Research in reinforcement learning, world modeling, and planning.
                 \end{itemize}

                The Collaboratory \hfill 2020- \\
                 {\sl Co-founder; Chief Science Officer}
                 \begin{itemize}  \itemsep -2pt %reduce space between items
                 \item Deep learning on language and graphs for scientific knowledge curation.
                 \item Led product strategy, design, and ML R\&D.
                 \item Admitted to Techstars class of 2021 ($<1\%$ applicants admitted).
                 \item Raised $>$\$2M (as of early 2022).
                 \end{itemize}

                 Comma.ai \hfill            2020 \\
		{\sl ML Research Intern}
                 \begin{itemize}  \itemsep -2pt %reduce space between items
                 \item Reinforcement learning for self-driving cars.
                 \end{itemize}

               Front Range Geosciences \hfill            2017-20 \\
                 {\sl Machine Learning Engineer}
                 \begin{itemize}  \itemsep -2pt %reduce space between items
                 \item Developed computer vision system for seismic data.
                 \end{itemize}
%\section{RESEARCH}
                 Center for Theory of Quantum Matter \hfill            2017 \\
		{\sl Research Assistant}
                 \begin{itemize}  \itemsep -2pt %reduce space between items
                 \item Studied quantum many-body localization under Floquet conditions.
                 \end{itemize}

                Mathematics Department, CU Boulder \hfill            2016 \\
		 {\sl Research Assistant}
                 \begin{itemize}  \itemsep -2pt %reduce space between items
                 \item Investigated knot-theoretic properties of topological quantum field theories.
                 \end{itemize}

                High Enery Particle Physics Group, Physics Department, CU Boulder  \hfill            2014-15 \\
		{\sl Research Assistant}
                 \begin{itemize}  \itemsep -2pt %reduce space between items
                 \item Monte Carlo simulations for the Deep Underground Neutrino Experiment.
                 \end{itemize}

                 \section{PUBLICATIONS}

                 {\sl DITTO: Offline Imitation Learning with World Models \hfill            $2022$}\\
		         In submission to ICLR 2023

                 {\sl Combining physics and deep learning to automatically \hfill            $2021$ \\ pick first breaks in the Permian Basin} \\
		First International Meeting for Applied Geoscience \& Energy

                {\sl Ein Liebesbrief an KataGo} \hfill 2020 \\
                Deutsche Go Zeitung, Ausgabe 4/2020

                {\sl Love Letter to KataGo, or:\hfill $2020$\\ Go AI Past, Present, and Future} \\
                American Go E-Journal

                {\sl DeepTrace: A breakthrough application of deep learning \hfill $2019$\\ to automate first break picking}  \\
                SEG 2019 Lenovo Thought Leadership Series

                {\sl Topology and Knot Theory} \hfill 2016 \\
                Course notes for CU Boulder special topics course: \\
                \textit{``Topology, Knot Theory, and their applications in Physics and Chemistry''}

                {\sl Secondary Particle Showers from Hadron Absorber Interactions} \hfill 2016 \\
                Deep Underground Neutrino Experiment (DUNE) Collaboration Documents


                \section{AWARDS}

                {\sl Research Studentship} \hfill Oxford, 2021 \\
                {\sl Stribic-Martin Scholarship} \hfill Boulder, 2017 \\
                        {\sl UROP Fellowship} \hfill Boulder, 2017 \\
                        {\sl Dawkins Fund Award} \hfill Oxford, 2016 \\
                        {\sl Gilman Scholarship} \hfill Oxford, 2016 \\
                        {\sl Esteemed Scholar Award} \hfill Boulder, 2014 \\




\end{resume}
\end{document}
